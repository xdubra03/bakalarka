\documentclass[10pt,xcolor=pdflatex]{beamer}
\usepackage{newcent}
\usepackage[utf8]{inputenc}
%\usepackage[czech]{babel}
\usepackage{hyperref}
\usepackage{fancyvrb}
\usetheme{FIT}

%%%%%%%%%%%%%%%%%%%%%%%%%%%%%%%%%%%%%%%%%%%%%%%%%%%%%%%%%%%%%%%%%%
\title[Stabilita]{Využitie strojového učenia na predikciu vplyvu mutácií na stabilitu proteínov}

\author[]{Juraj Ondrej Dúbrava}

\institute[]{Brno University of Technology, Faculty of Information Technology\\
Bo\v{z}et\v{e}chova 1/2. 612 66 Brno - Kr\'alovo Pole\\
xdubra03@fit.vutbr.cz}

%\date{January 1, 2016}
\date{\today}
%\date{} % bez data

%%%%%%%%%%%%%%%%%%%%%%%%%%%%%%%%%%%%%%%%%%%%%%%%%%%%%%%%%%%%%%%%%%

\begin{document}

\frame[plain]{\titlepage}

\begin{frame}\frametitle{Motivácia}
    \begin{itemize}
    	\item Vplyv mutácií na stabilitu
    	\item Neuspokojivé výsledky predikčných nástrojov 
    	\item Zlepšenie predikcie - návrh stabilnejších proteínov, účinnejších liečiv,...
    \end{itemize}
\end{frame}

\begin{frame}\frametitle{Proteíny}	
	\begin{itemize}
		\item základné stavebné prvky všetkých organizmov
		\item zabezpečujú množstvo funkcií
		\item proteín je tvorený reťazcom aminokyselín
		\item vlastnosti proteínov sú ovplyvnené mutáciami
	\end{itemize}
\end{frame}

\begin{frame}\frametitle{Stabilita proteínov}
	\begin{itemize}
		\item jedna z dôležitých vlastností proteínov, súvisí so stavom proteínu
		\item dôležitosť skúmania stability pre rôzne oblasti
		\item stabilné proteíny - lepšie zvládnutie nepriaznivých okolitých podmienok, vysokých teplôt,...
		\item pôsobenie mutácií na stabilitu, stabilizujúce vs. destabilizujúce mutácie 
		\item snaha o predikciu ich vplyvu, využitie strojového učenia
		
	\end{itemize}
\end{frame}

\begin{frame}\frametitle{Dataset}
	\begin{itemize}
		\item základ pre strojové učenie
		\item 1564 záznamov mutácií - 1255 destabilizújúcich, 309 stabilizujúcich
		\item 7 parametrov datasetu 
		\item výber metódy strojového učenia
		\item prvotné testovanie - nástroj WEKA, odskúšanie ponúkaných algoritmov
		\item najlepšie výsledky - metódy Random Forest, SVM
	\end{itemize}
\end{frame}

\begin{frame}\frametitle{Dataset}
\begin{center}
	\begin{tabular}{ | l | c | r | l| }
		\hline 
		Metóda & TP rate & FP rate & Accuracy \\ \hline
		Naive Bayes & 0,784 & 0,719 & 0,765 \\ \hline
		LibSVM &  0,786   & 0,706   & 0,766  \\ \hline
		SMO & 0,774 & 0,774 & 0,6\\ \hline
		DecisionTable & 0,774 & 0,774 & 0,6\\ \hline
		RandomForest & 0,793 & 0,692 & 0,797\\ \hline
		RandomTree & 0,793 & 0,574 & 0,766\\ \hline
		J48 & 0,774 & 0,626 & 0,74\\ \hline
		
	\end{tabular}
\end{center}
\end{frame}



\begin{frame}\frametitle{Implementácia}
	\begin{itemize}
		\item spôsob implementácie - Python skript
		\item použitie knižnice Scikit-learn na implementáciu algoritmov strojového učenia
		\item prvá fáza - odskúšanie metódy Random Forest
		\item neuspokojivé výsledky samostanej metódy
		\item implementácia a odskúšanie metódy SVM 
	\end{itemize}

\end{frame}

\begin{frame}\frametitle{Implementácia}
\begin{itemize}
	\item vylepšenie predikcie - použitie ensemble systému
	\item spojenie klasifikátorov - metódy Random Forest a SVM
	\item použitie pri nedostatku dát, realistickejší výsledok
	\item vytvorenie menších trénovacích podsád, bagging
	\item vylepšenie výpočtu parametra konzervovanosti - Felensteinov algoritmus
	\item presnosť predikcie - približne 74\%, korelácia 0.32
\end{itemize}

\end{frame}

\begin{frame}\frametitle{Záver}
\begin{itemize}
	\item analýza dátovej sady a jej použitie pre strojové učenie
	\item implementácia skriptu na výpočet parametrov
	\item implemtácia a otestovanie metód Random Forest a SVM
	\item použitie ensemble stratégie na vylepšenie presnosti
	\item výhoda rýchlej a dostatočne presnej predikcie
	\item možnosť použitia pre vytipovanie zaujímavých mutácií na ďaľšie skúmanie
\end{itemize}

\end{frame}




\bluepage{Ďakujem za pozornosť !}

\end{document}
